%! Author = asurk
%! Date = 23.02.2020

% Preamble
\documentclass[11pt]{article}

% Packages
\usepackage{amsmath}
\usepackage[T2A]{fontenc}
\usepackage[utf8]{inputenc}
\usepackage[margin=2cm]{geometry}
\usepackage{listings}
\usepackage{tikz}
\usetikzlibrary{shapes.geometric,arrows}

% Document
\begin{document}
    \input{titlepage.tex}

    \textbf{Цель работы:}
    научиться использовать итерационный численный метод решения систем линейных алгебраических уравнений.

    \textbf{Описание метода:}

    Пусть дано уравнение $A\cdot X=B$

    Преобразуем его к виду $X=C\cdot X+D$
    по формуле
    \[ c_{ij}=\left\{
    \begin{array}{cl}
        -\frac{a_{ij}}{a_{ii}} & i \neq j \\
        0 & i = j \\
    \end{array}
    \right. \]
    \[ d_i=\frac{b_i}{a_{ii}} \]

    Проверим диагональное преобладание в матрице $A$, т.е.
    \[ |a_{ii}|\geq \sum_{i\ne j} |a_{ij}| \]

    Тогда при условии диагонально преобладания в матрице $A$ можно получить сходящийся ряд
    $X_{n+1}=C\cdot X_n+D$, причем его предел -- это и есть решение изначальной системы

    При отсутствии диагонального преобладания попытаемся найти перестановку, в которой оно есть,
    методом коллапса волновой функции.
    В случае, если диагональное преобладание невозможно, сходимость ряда гарантировать нельзя.

    \lstset{
        extendedchars,
        numbers=left,
        basicstyle=\ttfamily\small,
        numberstyle=\ttfamily\small,
        inputencoding=cp1251
    }
    \lstinputlisting[title=\textbf{Листинг программы}]{main.py}

    \begin{center}
        \newpage
        \textbf{Блок-схема программы}

        \tikzstyle{startstop}=[rectangle,rounded corners,minimum width=3cm,minimum height=1cm,text centered,draw=black]
        \tikzstyle{io}=[trapezium,trapezium left angle=70,trapezium right angle=110,minimum width=3cm,minimum height=1cm,text centered,draw=black]
        \tikzstyle{process}=[rectangle,minimum width=3cm,minimum height=1cm,text centered,draw=black]
        \tikzstyle{decision}=[diamond,minimum width=3cm,minimum height=1cm,text centered,draw=black]
        \tikzstyle{arrow}=[->,>=stealth]
        \begin{tikzpicture}[node distance=2cm]
            \node (start) [startstop] {Начало};
            \node (in1) [io,below of=start] {Ввод $p,(A|B)$};
            \node (c_ij) [process,below of=in1,xshift=-5cm,yshift=-1cm] {$c_{ij}=-a_{ij}/a_{ii}$};
            \node (c_ii) [process,right of=c_ij,xshift=3cm] {$c_{ii}=0$};
            \node (d_i) [process,right of=c_ii,xshift=3cm] {$d_i=b_i/a_{ii}$};
            \node (x_i) [process,below of=d_i] {$x_i=0$};
            \node (X') [process,left of=x_i,xshift=-3cm] {$X'=C\cdot X+D$};
            \node (delta-x_i) [process,left of=X',xshift=-3cm] {$\Delta x_i=|x'_i-x_i|$};
            \node (X-X') [process,below of=delta-x_i,yshift=-1cm] {$X=X'$};
            \node (decision1) [decision,below of=X',yshift=-1cm] {$\max \Delta x_i < p?$};
            \node (out1) [io,below of=decision1,yshift=-1cm] {Вывод $X,\Delta X$};
            \node (stop) [startstop,below of=out1] {Конец};
%            \node (stop) [startstop,below of=c_ij] {Конец};
            \draw[arrow] (start) -- (in1);
            \draw[arrow] (in1) -- +(0,-1.5) -| (c_ij);
            \draw[arrow] (c_ij) -- (c_ii);
            \draw[arrow] (c_ii) -- (d_i);
            \draw[arrow] (d_i) -- (x_i);
            \draw[arrow] (x_i) -- (X');
            \draw[arrow] (X') -- (delta-x_i);
            \draw[arrow] (delta-x_i) -- (X-X');
            \draw[arrow] (X-X') -- (decision1);
            \draw[arrow] (decision1) -- node[anchor=west]{Нет} (X');
            \draw[arrow] (decision1) -- node[anchor=west]{Да} (out1);
            \draw[arrow] (out1) -- (stop);
        \end{tikzpicture}
    \end{center}

    \newpage
    \textbf{Пример}

    Точность -- $p=0.001$

    Система:
    \[
        \left\{
        \begin{array}{*{5}{r@{}lc}l}
              & x_1 & + &   & x_2 & + &   & x_3 & + &   & x_4 & + & 6 & x_5 & = & 1 \\
              & x_1 & + &   & x_2 & + &   & x_3 & + & 6 & x_4 & + &   & x_5 & = & 2 \\
              & x_1 & + &   & x_2 & + & 6 & x_3 & + &   & x_4 & + &   & x_5 & = & 3 \\
              & x_1 & + & 6 & x_2 & + &   & x_3 & + &   & x_4 & + &   & x_5 & = & 4 \\
            6 & x_1 & + &   & x_2 & + &   & x_3 & + &   & x_4 & + &   & x_5 & = & 5 \\
        \end{array}
        \right.
    \]

    Матрица системы:
    \[
        \left[
        \begin{array}{ccccc|c}
            1 & 1 & 1 & 1 & 6 & 1 \\
            1 & 1 & 1 & 6 & 1 & 2 \\
            1 & 1 & 6 & 1 & 1 & 3 \\
            1 & 6 & 1 & 1 & 1 & 4 \\
            6 & 1 & 1 & 1 & 1 & 5 \\
        \end{array}
        \right]
    \]

    Результат:
    \[
        X=\left[
        \begin{array}{r}
             0.70030449 \\
             0.50030449 \\
             0.30030449 \\
             0.10030449 \\
            -0.09969551 \\
      \end{array}
        \right],
        \Delta X=\left[
        \begin{array}{r}
            0.00076122 \\
            0.00076122 \\
            0.00076122 \\
            0.00076122 \\
            0.00076122 \\
        \end{array}
        \right],
        A\cdot X-B=\left[
        \begin{array}{r}
            0.00304488 \\
            0.00304488 \\
            0.00304488 \\
            0.00304488 \\
            0.00304488 \\
        \end{array}
        \right]
    \]

    Правильный ответ:
    \[
        X=\left[
        \begin{array}{r}
             0.7 \\
             0.5 \\
             0.3 \\
             0.1 \\
            -0.1 \\
        \end{array}
        \right];
        \Delta X=\left[
        \begin{array}{r}
            0.00030449 \\
            0.00030449 \\
            0.00030449 \\
            0.00030449 \\
            0.00030449 \\
        \end{array}
        \right]
    \]

    \textbf{Вывод:}
    метод простых итераций прост в реализации и легко преобразовывается к матричному виду,
    но требует подготовки в виде преобразования матрицы системы к виду с диагональным преобладанием,
    что не всегда возможно.
    В противном случае гарантировать сходимость метода нельзя.
\end{document}
